%!TEX program = xelatex

\documentclass[a4paper,UTF8]{ctexart}
\usepackage[margin=1.25in]{geometry}
\usepackage{amsmath}
\usepackage{amssymb}
\usepackage{changepage}
\usepackage{pgf}
\usepackage{tikz}
\usepackage{multicol}
\usetikzlibrary{arrows,automata}
% \usepackage[latin1]{inputenc}
\usepackage{verbatim}

\newcommand{\pro}[2]{\boldsymbol{#1} \rightarrow \boldsymbol{#2}}
\newcommand{\apro}[2]{\boldsymbol{#1} &\rightarrow \boldsymbol{#2}}

 \newcommand{\goto}{\vdash}
 \newcommand{\set}[1]{\{ #1 \}}
 \newcommand{\trans}[2]{\delta(#1)=(#2)}

\usepackage{color}
\usepackage{graphicx}
\usepackage{amssymb}
\usepackage{amsmath}
\usepackage{amsthm}
\usepackage{multirow}
\usepackage{pdfpages}
\usepackage{fontspec}                   % requires XeLaTeX

\theoremstyle{definition}
\newtheorem*{solution}{Solution}
\newtheorem*{prove}{Proof}

\setmainfont                    [Ligatures=TeX]{Times New Roman}
\parindent=0pt
\topskip=-200pt

\begin{document}
\title{PTC (Fall 2018) -- Assignment 3}
\author{徐翔哲 161250170}
\maketitle
\newpage

\section*{Problem 1}
Consider the (deterministic) Turing machine $M$ given by 
\[
M = (\{q_0, q_1, q_2\}, \{a, b\}, \{a, b, B\}, \delta, q_0, B, \{q_2\})
\]
which has exactly four transitions defined in it, as described below.
\begin{enumerate}
  \item $\delta(q_0, a) = (q_0, B, R)$
  \item $\delta(q_0, b) = (q_1, B, R)$
  \item $\delta(q_1, b) = (q_1, B, R)$
  \item $\delta(q_1, B) = (q_2, B, R)$
\end{enumerate}
Please answer the following questions:
\begin{enumerate}
  \item[a.]  Specify the execution trace of $M$ on the input string $abb$.
  \item[b.] Provide a regular expression for the language of the Turing machine.
  \item[c.]  Suppose we added the transition $\delta(q_1 ,a) = (q_0, B, R)$ to the above machine, provide a regular expression for the language of the resulting Turing machine.
\end{enumerate}

\subsection*{a}

$q_0aab \goto q_0ab \goto q_0b \goto q_1B \goto q_2B $

\subsection*{b}

$a^*b^+$

\subsection*{c}

$a^*b^+(a^+b^+)^*$


\section*{Problem 2}
Please design TM$'$s to decide following languages:
\begin{enumerate}
  \item[a.] $L_1 = \{1^{m} \times 1^{n} = 1^{mn}\ |\ m, n \in \mathbb{N}^{+}\}$ (e.g. $11 \times 111 = 111111 \in L_1$, but $1 \times 1 = 11 \notin L_1$)
  \item[b.] $L_2 = \{ww\ |\ w \in \{a,b\}^{*}\}$
\end{enumerate}
\subsection*{a}
First, let's define a TM $M_1$ which will convert $\times 1^n=1^{n'}$
 to $\times 1^n=1^{n'-n}$
where $n' \geq n > 0 $

\[M_1 = (\set{p_0,p_R,p_a,p_{a'},p_d,p_L,p_f,p_e},
\set{1,0,\times,=}
,\set{1,0,\times,=,B,a},\delta,p_0,B,\set{p_e})  \]
At the beginning , the head should at $\times$
Then we will enter $p_a$, which will change the first 1 to a 
when the head goes right.

$\trans{p_0, \times}{p_a,\times,R} $\\
$\trans{p_a,a}{p_a,a,R}$\\
$\trans{p_a,1}{p_R,a,R} $\\

The state $p_R$ moves the head to the first blank at the right of the input,
and then switches to state $p_d$, which will delete a 1.

$\trans{p_R,X}{p_R,X,R} $, where $X = \set{=,1}$\\
$\trans{p_R,B}{p_d,B,L} $\\
$\trans{p_d,1}{p_L,B,L}$\\
The state $p_L$ will moves to $\times$ and then enters $p_a'$

$\trans{p_L,X}{p_L,X,L}$, where $X=\set{1,=,a}$\\
$\trans{p_L,\times}{p_a,\times,R}$\\ 

Then the state $p_{a'}$ will act as $p_a$, change a 1 to a, then 
move to the right and delete a 1 ...... except that $p_{a'}$
will go to the state $p_f$ if it sees no 1 before the =,
under which condition $p_a$ would halt without accepting. 

$\trans{p_{a'},a}{p_{a'},a,R}$\\
$\trans{p_{a'},1}{p_R,a,R} $\\
$\trans{p_{a'},=}{p_f,=,L} $\\

The state $p_f$ will change all the a back to 1.

$\trans{p_f,a}{p_f,1,L}$\\
$\trans{p_f,\times}{p_e,\times,R}$

Then we construct the TM $M$ to describe the language $L_1$

\[M = (
  M_1.states \cup \set{q_0,q_R,q_{call},q_L,q_1,q_e,q_{check},q_f},
  \set{1,0,\times,=},
  \set{1,0,\times,=,B,a},
  M_1.\delta \cup \delta,
  q_0,B,\set{q_f} )\]

The state $q_0$ will remove the first 1, and goes to state $q_R$

$\trans{q_0,1}{q_R,B,R} $

The state $q_R$ will move to the $\times$ and then enters $q_{call}$

$\trans{q_R,1}{q_R,1,R}$\\
$\trans{q_R,\times}{q_{call},\times,L}$

The state $q_{call}$ will call the $M_1$

$\trans{q_call,1}{p_0,1,R}$

When the M returns from $M_1$, state $q_L$ will move the head
to the blank at the left end of the input and then enters $q_1$

$\trans{q_L,1}{q_L,1,L}$\\
$\trans{q_L,\times}{q_L,\times,L}$\\
$\trans{q_L,B}{q_1,B,R}$

$q_1$ acts just like $q_0$ except that when it finds no 1 before 
$\times$, it will switch to $q_e$

$\trans{q_1,1}{q_R,B,R}$\\
$\trans{q_1,\times}{q_e,\times,R}$\\

$q_e$ and $q_{check}$ will check whether the RHS of the = is blank.

$\trans{q_e,1}{q_e,1,R}$\\
$\trans{q_e,=}{q_{check},=,R}$\\
$\trans{q_{check},B}{q_f,B,L}$
 

\subsection*{b}

We will use multi-track TMs in this problem. 

First, define a TM $M_1$ which will compare whether two strings
are equal to each other. The beginning of the second string
will be marked as <x,c>, the beginning before the first 
string is marked as <z,B>, where c $\in$ \set{a,b}. Other 
related cells on the second track is initialized to *.

\[M_1(
  \set{q_0,q_c,q_a,q_b,q_{da},q_{db},q_L,q_f,q_e},
  \set{a,b},
  \set{a,b,x,*,z,B},
  \delta,
  q_0,
  B,
  \set{q_e}
)\]
$q_0$ will move the head to the end of the first string.
$\trans{q_0,<*,X>}{q_0,<*,X>,R}$, where $X \in \set{a,b}$

$\trans{q_0,<x,X>}{q_c,<x,X>,L}$, where $X \in \set{a,b}$

$q_c$ will remove a character from the tail of the first
string and remember this string in its state.
$\trans{q_c,<*,a>}{q_a,<*,B>,R}$

$\trans{q_c,<*,b>}{q_b,<*,B>,R}$

$\trans{q_a,<*,X>}{q_a,<*,X>,R}$

$\trans{q_b,<*,X>}{q_b,<*,X>,R}$

$q_a$ and $q_b$ will move the head to the end of the second 
string and remember to delete a or b respectively.

$\trans{q_a,<*,B>}{q_{da},<*,B>,L}$

$\trans{q_b,<*,B>}{q_{db},<*,B>,B}$

$q_{da}$ and $q_{db}$ will delete one a or one b respectively.

$\trans{q_{da},<*,a>}{q_L,<*,B>,L}$

$\trans{q_{da},<x,a>}{q_L,<x,B>,L}$

$\trans{q_{db},<*,b>}{q_L,<*,B>,L}$

$\trans{q_{db},<x,b>}{q_L,<x,B>,L}$

$q_L$ will move the head to the ending blanks of the 
first string.

$\trans{q_L,<*,X>}{q_L,<*,X>,L}$, where $X \in \set{a,b}$

$\trans{q_L,<x,X>}{q_c,<x,X>,L}$, where $X \in \set{a,b}$

$q_c$ will find the next char to remove,
when it meets z on the second track, the TM goes
to a new state $q_f$

$\trans{q_c,<*,B>}{q_c,<*,B>,L}$

$\trans{q_c,<z,B>}{q_f,<z,B>,R}$

$q_f$ will check whether the second string 
has been reduced to blanks, if so, it will 
change to the accepting state$q_e$.

$\trans{q_f,<*,B>}{q_f,<*,B>,R}$

$\trans{q_f,<x,B>}{q_e,<x,B>,R}$

Then we define the TM M for language $L_2$.

$M(
  M_1.states \cup \set{p_0,p_1,p_{odd},p_{even},p_{back},p_{fwd},p_R,p_{last},p_{next},p_{markZ}},
  \set{a,b}, 
  \set{a,b,x,*,t,z,B},\\
  M_1.\delta \cup \delta,
  p_0,
  B,
  \set{q_e}
)$

M will accept the empty strings.

$\trans{p_0,<B,B>}{q_e,<B,B>,R}$

Firstly, marked the first blank at the left of the input as x.

$\trans{p_0,<B,X>}{p_1,<B,X>,L}$

$\trans{p_1,<B,B>}{p_{odd},<x,B>,R}$

If this is the odd indexed(start from 1) char from the input, then the state will change to 
even and vice versa.

$\trans{p_{odd},<B,X>}{p_{even},<*,X>,R}$, where $X \in \set{a,b}$

$\trans{p_{even},<B,X>}{p_{back},<t,X>,L}$, where $X \in \set{a,b}$

When M makes every two steps right, it will move x forward for one step. Then when the 
first blank at right is met, the x will be moved to the end of the first string.

$\trans{p_{back},<*,X>}{p_{back},<*,X>,L}$, where $X \in \set{a,b}$

$\trans{p_{back},<x,X>}{p_{fwd},<*,X>,R}$, where $X \in \set{a,b,B}$

$\trans{p_{fwd},<*,X>}{p_R,<x,X>,R}$, where $X \in \set{a,b}$

$\trans{p_R,<*,X>}{p_R,<*,X>,R}$, where $X \in \set{a,b}$

$\trans{p_R,<t,X>}{p_{odd},<*,X>,R}$, where $X \in \set{a,b}$

Then we move x forward for one more step.

$\trans{p_{odd},<B,B>}{p_{last},<*,B>,R}$

$\trans{p_{last},<*,X>}{p_{last},<*,X>,L}$, where $X \in \set{a,b}$

$\trans{p_{last},<x,X>}{p_{next},<*,X>,R}$, where $X \in \set{a,b}$

$\trans{p_{next},<*,X>}{p_{markZ},<x,X>,L}$, where $X \in \set{a,b}$

$\trans{p_{markZ},<*,X>}{p_{markZ},<*,X>,L}$, where $X \in \set{a,b}$

Finally, we marked the blank at the left of the input as z, and call $M_1$.

$\trans{p_{markZ},<*,B>}{q_0,<z,B>,R}$



\section*{Problem 3}
A \textit{useless state} in a Turing machine is one that is never 
entered on any input string. Consider the problem of determining whether a state in a 
Turing machine is useless. Formulate this problem as a language and show it is 
decidable or undecidable. (\textbf{Hint}: consider the language $E_{\mathrm{TM}}$)
\\

Construct a TM M' as follows.
Given any input s, if s can be interpreted as $<M,w>$, where M is a TM, w is a string,
then we input it to the algorithm for ATM. If ATM accepts $<M,w>$, M' will enter a state $QF$, where
$QF \notin$ the state set of ATM.

 It's obvious that the TM M' will enter the state $QF$ if and only if the 
ATM accepts $<M,w>$. So if we can decide whether or not $QF$ is a useless state, then we can decide ATM. Since
ATM is undecidable, so useless state is undecidable.


\section*{Problem 4}
Show that the following questions are decidable:
\begin{enumerate}
  \item[a.] The set $L$ of codes for TM’s $M$ such that, when started with the blank tape will eventually write some nonblank symbol on its tape. (\textbf{Hint}: If $M$ has $m$ states, consider the first $m$ transitions that it makes)
  \item[b.] The set $L$ of codes for TM’s that never make a move left on any input.
  \item[c.] The set $L$ of pairs $(M,w)$ such that TM $M$, started with input $w$, never scans any tape cell more than once.
\end{enumerate}

\subsection*{a}
Let's construct a directed graph(N,E,P), where N is the set of all the nodes, 
E is the set of edges and P is a set for some special nodes. For each transition
$\trans{p,X}{q,Y,D}$, if X is B, we add a node pXq. Also, if Y is not B, then 
we also add this node to the set P. Specially, we add a node $p_0Bp_0$, where $p_0$ is 
the start state of M. 
For each pair of nodes, say, $p_1Bq_1$, $p_2Bq_2$, 
if $q_1$=$p_2$ and $p_2$ is not a final state , add an edge ($q_1$, $p_2$). \\

Then we traverse the sub-graph which can be access from $p_0Bp_0$. When we meet any node in P, then 
we accept this M. Otherwise, if we never meet any node in P after visiting all the reachable 
node, we reject this M.


\subsection*{b}
Let's construct a directed graph(N,E,P), where N is the set of all the nodes, 
E is the set of edges and P is a set for some special nodes. Specially, we add a node $p_0Bp_0$,
 where $p_0$ is 
the start state of M.  For each transition
$\trans{p,X}{q,Y,D}$, we add a node pXq. Also, if D is L, then 
we also add this node to the set P. For each pair of nodes, say, $p_1Xq_1$, $p_2Yq_2$, 
if $q_1$=$p_2$ and $p_2$ is not a final state, add an edge ($q_1$, $p_2$). \\

Then we traverse the sub-graph which can be access from $p_0Bp_0$. When we meet any node in P, then 
we reject this M. Otherwise, if we never meet any node in P after visiting all the reachable 
node, we accept this M.


\subsection*{c}
Suppose that the ID of M is $q_0w$ at the beginning.

\subsubsection*{case 1}
If there exists
$\trans{q_0,c}{q',Y,L}$, where c is the first character of w, q' is an arbitrary
state and Y is an arbitrary tape character, then we construct a graph(N,E,P) as follows.
Forall $\trans{p,B}{q,Y,D}$, add a node pBq in N. If D is R, also add this node in P.
Specially, we add a node $q_0Xq'$.  For each pair of nodes, say, $p_1Xq_1$, $p_2Yq_2$, 
if $q_1$=$p_2$ and $p_2$ is not a final state, add an edge ($q_1$, $p_2$). \\

Then we traverse the sub-graph which can be access from $q_0Bq'$. If we meet any node in P,
we reject this (M,w).

\subsubsection*{case2}
If there exists $\trans{q_0,c}{q',Y,R}$, where c is the first character of w, q' is an arbitrary
state and Y is an arbitrary tape character, then M cannot move any step to left.
Suppose the length of w is n. Then we simulate M for n steps.  
If M moves to left in any step, we reject this (M,w).
If M halts within n steps not moving to left, then we accept.

\subsubsection*{case3}
If M doesn't halt within n steps, the head must point to the B at the right of w,
 suppose the state now is $q_x$.

We construct a graph(N,E,P) as follows.
Forall $\trans{p,B}{q,Y,D}$, add a node pBq in N. If D is L, also add this node in P.
Specially, we add a node $q_xBq_x$.  For each pair of nodes, say, $p_1Xq_1$, $p_2Yq_2$, 
if $q_1$=$p_2$ and $p_2$ is not a final state, add an edge ($q_1$, $p_2$). \\


Then we traverse the sub-graph which can be access from $q_xBq_x$. If we meet any node in P,
we reject this (M,w). Otherwise, we accept this (M,w).

\newpage

\section*{Problem 5}
If a pushdown automaton has $k$ stacks, we call it $k$−PDA. Clearly, 0−PDA is NFA, 1−PDA is PDA, and 1−PDA is more powerful than 0−PDA.
\begin{enumerate}
  \item[1.] What is the difference between the express ability of 2-PDA and 1-PDA. Please clarify your argument. Prove the (un)equivalence.
  \item[2.] How about 3-PDA and 2-PDA.
\end{enumerate}

\subsection*{1}
It's obvious that 2-PDA and 1-PDA are not equivalence. Since 2-PDA can accept languages like $a^nb^nc^n$
while 1-PDA cannot. Also, it's trivial to show that 2-PDA can express any language 1-PDA can accept.
So 2-PDA is more powerful than 1-PDA.

\subsection*{2}
3-PDA and 2-PDA are equivalence.
I'll show that both 3-PDA and 2-PDA are equivalence to TM.

Firstly, I'll prove that every language accepted by 2-PDA can be accepted by TM.

We choose a 2-tape TM. The first tape contains the input, and the second tape is a simulation for the stack.
For any transition $\delta'(q,a,X)=(p,Y)$, supposed that the length of 
Y is k, we will get these transitions for the TM:

If Y=$\epsilon$, for the first tape: $\trans{q,a}{p,a,R}$; 
for the second tape: $\trans{q,X}{p,B,L}$

If Y$\ne \epsilon$, for the first tape:

$\trans{q,a}{p,a,R}$
for the second tape: $\trans{q,X}{q_{k-1},Y[k-1],R}$

Also, for each integer i from 0 to k-2, exists $\trans{q_{i+1},B}{q_{i},Y[i],R}$

finally, add $\trans{q_0,B}{p,B,L}$

Similarly, we can simulate 3 stack using a 3-tape TM.

Now we have shown that any language accepted by 2-PDA and 3-PDA can be accepted by a TM.

Then we'll prove that 2-PDA can simulate a TM.

At the beginning,
we just push all the input into stack A, and pop the character from stack A and push this character into stack B
one by one, until we meet the stack-bottom marker of A. Then the stack B holds characters at the right 
of the TM head, the stack A holds characters at the left of the TM head.

For each transition in TM noted as $\delta'(p,X)=(q,Y,D)$: 

If D is L, add $\trans{p,\epsilon, A, X}{q,\epsilon,AY}$, where A is any symbol at the top of A except 
for the stack-bottom marker of A, noted as $\perp_A$.

When the top of A is $\perp_A$, add $\trans{p,\epsilon,\perp_A, X}{q,\perp_A,BY}$

If D is R, add $\trans{p,\epsilon, A, X}{q,YA,\epsilon}$.

In addition, if X is B(the blank symbol in M), also add $\trans{p,\epsilon, A, \perp_B}{q,YA,\perp_B}$ (D=R)
or $\trans{p,\epsilon,A, \perp_B}{q,\epsilon,AY\perp_B}$(D=L \& A $\neq$ $\perp_A$) and 
$\trans{p,\epsilon,\perp_A, \perp_B}{q,\perp_A,BY\perp_B}$(D=L \& A = $\perp_A$); 

Finally, change all the occurrences like $B\perp_B$, $B\perp_A$ to $\perp_B$, $\perp_A$ respectively.

Now we have shown that 2-PDA can simulate a TM. Since 3-PDA can simulate 2-PDA by using only 2 stack,
3-PDA can simulate a TM. Then both 2-PDA and 3-PDA are equivalent to TM, which means they are equivalent.



\section*{Problem 6}
Suppose we have an encoding of context-free grammars using some finite alphabet. Consider the following two languages:
\begin{enumerate}
  \item[1.] $L_1$ = \big\{$(G,A,B)$\ |\ $G$ is a (coded) CFG, $A$ and $B$ are (coded) varibles of $G$, and the sets of terminal strings derived from $A$ and $B$ are the same\big\}.
  \item[2.] $L_2$ = \big\{$(G_1,G_2)$\ |\ $G_1$ and $G_2$ are (coded) CFG$'$s, and $L(G_1) = L(G_2)$\big\}.
\end{enumerate}
Answer the following questions:
\begin{enumerate}
  \item[a.] Show that $L_1$ is polynomial-time reducible to $L_2$.
  \item[b.] Show that $L_2$ is polynomial-time reducible to $L_1$.
\end{enumerate}
\subsection*{a}
Just copy the language G twice, but mark A as the start symbol of $G_1$,
mark B as the start symbol of $G_2$. The copy action will consume just O(n) time,
and the modification will consume constant time. So the reduction is polynomial-time.
Also, we should prove that when $L_2$ accepts, $L_1$ accepts and when $L_2$ rejects, $L_1$ rejects.

If $L_2$ accepts, that means the terminal strings derived from the start symbol of $G_1$ and $G_2$
are equivalent. Since $G_1$ and $G_2$ share the same productions in G, then they will surely derivate
the same terminal strings when they're two variables in G.

If $L_2$ rejects, we can assume that there is a string w, which is in $L(G_1)$ but is not in $L(G_2)$,
then variable A in G can derivate the terminal string w but variable B cannot. Then (G,A,B) should be rejected.

\subsection*{b}
Suppose that the start symbol of $G_1$ and $G_2$ are $S_1$, $S_2$ respectively.
Then construct G as $S\rightarrow S_1|S_2$. Before we copy productions from $G_1$ and $G_2$, we 
just rename variables in $G_1$ such that variables in $G_1$ are disjoint from variables in $G_2$.
Then we get an instance of $L_1$ which will be $(G,S_1,S_2)$. It's obvious that the reduction takes
O(n) time. Now we're going to prove that when $L_1$ accepts, so does $L_2$,
 and when $L_1$ rejects, so does $L_1$.


If $L_1$ accepts, that means $S_1$ and $S_2$ can derivate the exact same terminal string set. 
Also, notice that all the production they can use are belongs to $G_1$ and $G_2$ respectively since
we have guarantee that the variables in $G_1$ are disjoint with variables in $G_2$. So for each terminal
string they can derivate in G, they can also derivate it in $G_1$ or $G_2$. So we can 
conclude that $L(G_1)=L(G_2)$.

If $L_1$ rejects, we can assume that there is a terminal string w that belongs to 
the terminal strings of $S_1$ but not belongs to the terminal strings of $S_2$. It's obvious that
$S_1$ can derivate w in $G_1$. So $L(G_1) \neq L(G_2)$.

\section*{Problem 7}
As classes of languages, $\mathcal{P}$ and $\mathcal{NP}$ each have certain closure properties.
 Prove or disprove that $\mathcal{P}$ and $\mathcal{NP}$ are closed under each of the following operations:
\begin{enumerate}
  \item[a.] Union.
  \item[b.] Concatenation.
  \item[c.] Complementation.
\end{enumerate}
\subsection*{a}
Both P and NP are closed under union.

P: For any P languages $P_1$ and $P_2$, noted the related TM as $M_1$, $M_2$. Construct 
a two-tape TM M' as follows:

Suppose that we have a subroutine S that will copy the input from tape 1 to tape 2 in O(n) time.
The accepting states, alphabet, tape symbols of M' are the union of
 their counterparts in $M_1$, $M_2$ and S.

% The states of M will be the union of states in $M_1$, $M_2$, S and $q_s$. $q_s$ will initialize the 
% running environment for $M_2$.

The start state is the start state of S

1. For each transition in $M_1$, noted as $\trans{p,X}{q,Y,D}$,
 add $\trans{p,X,*}{q,Y,*,D,S}$, where * is any tape symbol on tape 2.
Similarly, for each transition in $M_2$, noted as $\trans{p,X}{q,Y,D}$,
 add $\trans{p,*, X}{q,*,Y,S, D}$, where * is any tape symbol on tape 1.

2. The transitions will also include transitions that will copy the input to tape 2.

3. Also, for every  
non-accepting state p and every tape symbols c, if $(p,c)$ is not defined in the 
transitions of $M_1$, add $\trans{p,c,*}{q_2,c,*,R,S}$, where $q_2$ is the start state of $M_2$.

Since $M_1$ and $M_2$ will halt within polynomial time, the maximum time M' consumes will in 
polynomial times.

NP: Similarly, if $M_1$ and $M_2$ are NTMs halt in polynomial time,
 then M' will be a NTM. The maximum time M' consumes will still in 
polynomial times.

\subsection*{b}
P and NP are closed under concatenation. 

We'll reuse the notations of problem a.
P: 

The input can be divided as two substrings with length [0,n],[2,n-2].....[n,0]

Construct a TM with 2n tapes. Group the tapes into n group.
For each group, there will be a division of the input string with the 
first part on one tape and the second one for another.

The division will takes O($n^2$) time since we will copy the string of length n into n groups.

Then for each group, we use $M_1$ and $M_2$ to decide them respectively. If both $M_1$ and $M_2$ are 
accepted, then we can say $M_1$ and $M_2$ accepts this group. If any group is accepted, then M' accepts 
the input, otherwise rejects.

The sum of the decide time for one group should be polynomial. For n groups, it will still be polynomial.

So M' can decide the input in polynomial time.

NP: The NTM M' will choose one division, 
then we use $M_1$ and $M_2$ to decide these two parts respectively. 
If both $M_1$ and $M_2$ are 
accepted, then NTM accepts this input.

If under no circumstances M' can accept this input, then M' will reject it.
The decision will be made in polynomial time.


\subsection*{c}
P and NP are closed under complementation. 

P:
First, copy the input to another tape. (O(n))

Second, see whether $M_1$ accepts this input. If not, M' rejects this input. (polynomial)

(If $M_1$ accepts), third, see whether $M_2$ accepts this input,
if so, the M' rejects this input, otherwise accepts. (polynomial)

So the total time would be polynomial.

NP:
The time analysis for NTM is similar to that for TM.


\end{document} 