%!TEX program = xelatex

\documentclass[a4paper,UTF8]{ctexart}
\usepackage[margin=1.25in]{geometry}
\usepackage{amsmath}
\usepackage{amssymb}
\usepackage{changepage}
\usepackage{pgf}
\usepackage{tikz}
\usepackage{multicol}
\usetikzlibrary{arrows,automata}
% \usepackage[latin1]{inputenc}
\usepackage{verbatim}

\newcommand{\pro}[2]{\boldsymbol{#1} \rightarrow \boldsymbol{#2}}
\newcommand{\apro}[2]{\boldsymbol{#1} &\rightarrow \boldsymbol{#2}}

 \newcommand{\goto}{\vdash}
 \newcommand{\set}[1]{\{ #1 \}}
 \newcommand{\trans}[2]{\delta(#1)=(#2)}

\usepackage{color}
\usepackage{graphicx}
\usepackage{amssymb}
\usepackage{amsmath}
\usepackage{amsthm}
\usepackage{multirow}
\usepackage{pdfpages}
\usepackage{fontspec}                   % requires XeLaTeX

\theoremstyle{definition}
\newtheorem*{solution}{Solution}
\newtheorem*{prove}{Proof}

\setmainfont                    [Ligatures=TeX]{Times New Roman}
\parindent=0pt
\topskip=-200pt

\begin{document}
\title{PTC (Fall 2018) -- Assignment 3}
\author{徐翔哲 161250170}
\maketitle
\newpage

\section*{Problem 1}
Consider the (deterministic) Turing machine $M$ given by 
\[
M = (\{q_0, q_1, q_2\}, \{a, b\}, \{a, b, B\}, \delta, q_0, B, \{q_2\})
\]
which has exactly four transitions defined in it, as described below.
\begin{enumerate}
  \item $\delta(q_0, a) = (q_0, B, R)$
  \item $\delta(q_0, b) = (q_1, B, R)$
  \item $\delta(q_1, b) = (q_1, B, R)$
  \item $\delta(q_1, B) = (q_2, B, R)$
\end{enumerate}
Please answer the following questions:
\begin{enumerate}
  \item[a.]  Specify the execution trace of $M$ on the input string $abb$.
  \item[b.] Provide a regular expression for the language of the Turing machine.
  \item[c.]  Suppose we added the transition $\delta(q_1 ,a) = (q_0, B, R)$ to the above machine, provide a regular expression for the language of the resulting Turing machine.
\end{enumerate}

\subsection*{a}

$q_0aab \goto q_0ab \goto q_0b \goto q_1B \goto q_2B $

\subsection*{b}

$a^*b^+$

\subsection*{c}

$a^*b^+(a^+b^+)^*$


\section*{Problem 2}
Please design TM$'$s to decide following languages:
\begin{enumerate}
  \item[a.] $L_1 = \{1^{m} \times 1^{n} = 1^{mn}\ |\ m, n \in \mathbb{N}^{+}\}$ (e.g. $11 \times 111 = 111111 \in L_1$, but $1 \times 1 = 11 \notin L_1$)
  \item[b.] $L_2 = \{ww\ |\ w \in \{a,b\}^{*}\}$
\end{enumerate}
\subsection*{a}
First, let's define a TM $M_1$ which will convert $\times 1^n=1^{n'}$
 to $\times 1^n=1^{n'-n}$
where $n' \geq n > 0 $

\[M_1 = (\set{p_0,p_R,p_a,p_{a'},p_d,p_L,p_f,p_e},
\set{1,0,\times,=}
,\set{1,0,\times,=,B,a},\delta,p_0,B,\set{p_e})  \]
At the beginning , the head should at $\times$
Then we will enter $p_a$, which will change the first 1 to a 
when the head goes right.

$\trans{p_0, \times}{p_a,\times,R} $\\
$\trans{p_a,a}{p_a,a,R}$\\
$\trans{p_a,1}{p_R,a,R} $\\

The state $p_R$ moves the head to the first blank at the right of the input,
and then switches to state $p_d$, which will delete a 1.

$\trans{p_R,X}{p_R,X,R} $, where $X = \set{=,1}$\\
$\trans{p_R,B}{p_d,B,L} $\\
$\trans{p_d,1}{p_L,B,L}$\\
The state $p_L$ will moves to $\times$ and then enters $p_a'$

$\trans{p_L,X}{p_L,X,L}$, where $X=\set{1,=,a}$\\
$\trans{p_L,\times}{p_a,\times,R}$\\ 

Then the state $p_{a'}$ will act as $p_a$, change a 1 to a, then 
move to the right and delete a 1 ...... except that $p_{a'}$
will go to the state $p_f$ if it sees no 1 before the =,
under which condition $p_a$ would halt without accepting. 

$\trans{p_{a'},a}{p_{a'},a,R}$\\
$\trans{p_{a'},1}{p_R,a,R} $\\
$\trans{p_{a'},=}{p_f,=,L} $\\

The state $p_f$ will change all the a back to 1.

$\trans{p_f,a}{p_f,1,L}$\\
$\trans{p_f,\times}{p_e,\times,R}$

Then we construct the TM $M$ to describe the language $L_1$

\[M = (
  M_1.states \cup \set{q_0,q_R,q_{call},q_L,q_1,q_e,q_{check},q_f},
  \set{1,0,\times,=},
  \set{1,0,\times,=,B,a},
  M_1.\delta \cup \delta,
  q_0,B,\set{q_f} )\]

The state $q_0$ will remove the first 1, and goes to state $q_R$

$\trans{q_0,1}{q_R,B,R} $

The state $q_R$ will move to the $\times$ and then enters $q_{call}$

$\trans{q_R,1}{q_R,1,R}$\\
$\trans{q_R,\times}{q_{call},\times,L}$

The state $q_{call}$ will call the $M_1$

$\trans{q_call,1}{p_0,1,R}$

When the M returns from $M_1$, state $q_L$ will move the head
to the blank at the left end of the input and then enters $q_1$

$\trans{q_L,1}{q_L,1,L}$\\
$\trans{q_L,\times}{q_L,\times,L}$\\
$\trans{q_L,B}{q_1,B,R}$

$q_1$ acts just like $q_0$ except that when it finds no 1 before 
$\times$, it will switch to $q_e$

$\trans{q_1,1}{q_R,B,R}$\\
$\trans{q_1,\times}{q_e,\times,R}$\\

$q_e$ and $q_{check}$ will check whether the RHS of the = is blank.

$\trans{q_e,1}{q_e,1,R}$\\
$\trans{q_e,=}{q_{check},=,R}$\\
$\trans{q_{check},B}{q_f,B,L}$
 

\subsection*{b}

We will use multi-track TMs in this problem. 

First, define a TM $M_1$ which will compare whether two strings
are equal to each other. The beginning of the second string
will be marked as <x,c>, the beginning before the first 
string is marked as <z,B>, where c $\in$ \set{a,b}. Other 
related cells on the second track is initialized to *.

\[M_1(
  \set{q_0,q_c,q_a,q_b,q_{da},q_{db},q_L,q_f,q_e},
  \set{a,b},
  \set{a,b,x,*,z,B},
  \delta,
  q_0,
  B,
  \set{q_e}
)\]
$q_0$ will move the head to the end of the first string.
$\trans{q_0,<*,X>}{q_0,<*,X>,R}$, where $X \in \set{a,b}$

$\trans{q_0,<x,X>}{q_c,<x,X>,L}$, where $X \in \set{a,b}$

$q_c$ will remove a character from the tail of the first
string and remember this string in its state.
$\trans{q_c,<*,a>}{q_a,<*,B>,R}$

$\trans{q_c,<*,b>}{q_b,<*,B>,R}$

$\trans{q_a,<*,X>}{q_a,<*,X>,R}$

$\trans{q_b,<*,X>}{q_b,<*,X>,R}$

$q_a$ and $q_b$ will move the head to the end of the second 
string and remember to delete a or b respectively.

$\trans{q_a,<*,B>}{q_{da},<*,B>,L}$

$\trans{q_b,<*,B>}{q_{db},<*,B>,B}$

$q_{da}$ and $q_{db}$ will delete one a or one b respectively.

$\trans{q_{da},<*,a>}{q_L,<*,B>,L}$

$\trans{q_{da},<x,a>}{q_L,<x,B>,L}$

$\trans{q_{db},<*,b>}{q_L,<*,B>,L}$

$\trans{q_{db},<x,b>}{q_L,<x,B>,L}$

$q_L$ will move the head to the ending blanks of the 
first string.

$\trans{q_L,<*,X>}{q_L,<*,X>,L}$, where $X \in \set{a,b}$

$\trans{q_L,<x,X>}{q_c,<x,X>,L}$, where $X \in \set{a,b}$

$q_c$ will find the next char to remove,
when it meets z on the second track, the TM goes
to a new state $q_f$

$\trans{q_c,<*,B>}{q_c,<*,B>,L}$

$\trans{q_c,<z,B>}{q_f,<z,B>,R}$

$q_f$ will check whether the second string 
has been reduced to blanks, if so, it will 
change to the accepting state$q_e$.

$\trans{q_f,<*,B>}{q_f,<*,B>,R}$

$\trans{q_f,<x,B>}{q_e,<x,B>,R}$

Then we define the TM M for language $L_2$.

$M(
  M_1.states \cup \set{p_0,p_1,p_{odd},p_{even},p_{back},p_{fwd},p_R,p_{last},p_{next},p_{markZ}},
  \set{a,b}, 
  \set{a,b,x,*,t,z,B},\\
  M_1.\delta \cup \delta,
  p_0,
  B,
  \set{q_e}
)$

M will accept the empty strings.

$\trans{p_0,<B,B>}{q_e,<B,B>,R}$

Firstly, marked the first blank at the left of the input as x.

$\trans{p_0,<B,X>}{p_1,<B,X>,L}$

$\trans{p_1,<B,B>}{p_{odd},<x,B>,R}$

If this is the odd indexed(start from 1) char from the input, then the state will change to 
even and vice versa.

$\trans{p_{odd},<B,X>}{p_{even},<*,X>,R}$, where $X \in \set{a,b}$

$\trans{p_{even},<B,X>}{p_{back},<t,X>,L}$, where $X \in \set{a,b}$

When M makes every two steps right, it will move x forward for one step. Then when the 
first blank at right is met, the x will be moved to the end of the first string.

$\trans{p_{back},<*,X>}{p_{back},<*,X>,L}$, where $X \in \set{a,b}$

$\trans{p_{back},<x,X>}{p_{fwd},<*,X>,R}$, where $X \in \set{a,b,B}$

$\trans{p_{fwd},<*,X>}{p_R,<x,X>,R}$, where $X \in \set{a,b}$

$\trans{p_R,<*,X>}{p_R,<*,X>,R}$, where $X \in \set{a,b}$

$\trans{p_R,<t,X>}{p_{odd},<*,X>,R}$, where $X \in \set{a,b}$

Then we move x forward for one more step.

$\trans{p_{odd},<B,B>}{p_{last},<*,B>,R}$

$\trans{p_{last},<*,X>}{p_{last},<*,X>,L}$, where $X \in \set{a,b}$

$\trans{p_{last},<x,X>}{p_{next},<*,X>,R}$, where $X \in \set{a,b}$

$\trans{p_{next},<*,X>}{p_{markZ},<x,X>,L}$, where $X \in \set{a,b}$

$\trans{p_{markZ},<*,X>}{p_{markZ},<*,X>,L}$, where $X \in \set{a,b}$

Finally, we marked the blank at the left of the input as z, and call $M_1$.

$\trans{p_{markZ},<*,B>}{q_0,<z,B>,R}$



\section*{Problem 3}
A \textit{useless state} in a Turing machine is one that is never entered on any input string. Consider the problem of determining whether a state in a Turing machine is useless. Formulate this problem as a language and show it is decidable or undecidable. (\textbf{Hint}: consider the language $E_{\mathrm{TM}}$)
\begin{solution}
\end{solution}
\newpage

\section*{Problem 4}
Show that the following questions are decidable:
\begin{enumerate}
  \item[a.] The set $L$ of codes for TM’s $M$ such that, when started with the blank tape will eventually write some nonblank symbol on its tape. (\textbf{Hint}: If $M$ has $m$ states, consider the first $m$ transitions that it makes)
  \item[b.] The set $L$ of codes for TM’s that never make a move left on any input.
  \item[c.] The set $L$ of pairs $(M,w)$ such that TM $M$, started with input $w$, never scans any tape cell more than once.
\end{enumerate}
\begin{prove}
\end{prove}
\newpage

\section*{Problem 5}
If a pushdown automaton has $k$ stacks, we call it $k$−PDA. Clearly, 0−PDA is NFA, 1−PDA is PDA, and 1−PDA is more powerful than 0−PDA.
\begin{enumerate}
  \item[1.] What is the difference between the express ability of 2-PDA and 1-PDA. Please clarify your argument. Prove the (un)equivalence.
  \item[2.] How about 3-PDA and 2-PDA.
\end{enumerate}
\begin{solution}
\end{solution}
\newpage

\section*{Problem 6}
Suppose we have an encoding of context-free grammars using some finite alphabet. Consider the following two languages:
\begin{enumerate}
  \item[1.] $L_1$ = \big\{$(G,A,B)$\ |\ $G$ is a (coded) CFG, $A$ and $B$ are (coded) varibles of $G$, and the sets of terminal strings derived from $A$ and $B$ are the same\big\}.
  \item[2.] $L_2$ = \big\{$(G_1,G_2)$\ |\ $G_1$ and $G_2$ are (coded) CFG$'$s, and $L(G_1) = L(G_2)$\big\}.
\end{enumerate}
Answer the following questions:
\begin{enumerate}
  \item[a.] Show that $L_1$ is polynomial-time reducible to $L_2$.
  \item[b.] Show that $L_2$ is polynomial-time reducible to $L_1$.
\end{enumerate}
\begin{prove}
\end{prove}
\newpage

\section*{Problem 7}
As classes of languages, $\mathcal{P}$ and $\mathcal{NP}$ each have certain closure properties. Prove or disprove that $\mathcal{P}$ and $\mathcal{NP}$ are closed under each of the following operations:
\begin{enumerate}
  \item[a.] Union.
  \item[b.] Concatenation.
  \item[c.] Complementation.
\end{enumerate}
\begin{prove}
\end{prove}
\end{document} 